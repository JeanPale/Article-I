\section{Multiple clusters}

Within the Mie theory following multiple clusters spatial configuration considered: simple cubic lattice with 4 edge nodes and different length of the edge of unit cell $b : \{\lambda_{10},\:2\lambda_{10},\:3\lambda_{10}\}$. Cluster radius $a = 6.37$ nm. The CELES software package was used for modeling \cite{celes}.

For the case of $b = \lambda_{10}$ we can see efficient scattering by facets of the spatial lattice at angles $\approx 150^{\circ}$, $-150^{\circ}$ relative to the wave vector direction (\autoref{multi_sph_b1:image}d). Most of the field is localized in the area of clusters.

For $b = 2\lambda_{10}$ and angle of incidence $\approx 30^{\circ}$ there is a slight increase in scattering at small angles (\autoref{multi_sph_b2:image}d).

For $b = 3\lambda_{10}$ due to the increased rarefaction between clusters allows us to get rid of strong reflection, which can be seen from the far-field (\autoref{multi_sph_b3:image}d).

\begin{figure}[H]
    (a)\:\subimg[components/img/celes/64sph_b83nm_l83nm_0deg_near]{0.4\textwidth}
    (b)\:\subimg[components/img/celes/64sph_b83nm_l83nm_0deg_far]{0.4\textwidth}
    \\(c)\:\subimg[components/img/celes/64sph_b83nm_l83nm_45deg_near]{0.4\textwidth}
    (d)\:\subimg[components/img/celes/64sph_b83nm_l83nm_45deg_far]{0.4\textwidth}
    \caption{$10$-th harmonic scattering by multiple clusters. $b = \lambda_{10}$; a, b --- normal incidence; c, d --- incidence at $45^{\circ}$ angle; a, c --- near-field; b, d --- far-field.}
    \label{multi_sph_b1:image}
\end{figure}

\begin{figure}[H]
    (a)\:\subimg[components/img/celes/64sph_b166nm_l83nm_0deg_near]{0.4\textwidth}
    (b)\:\subimg[components/img/celes/64sph_b166nm_l83nm_0deg_far]{0.4\textwidth}
    \\(c)\:\subimg[components/img/celes/64sph_b166nm_l83nm_30deg_near]{0.4\textwidth}
    (d)\:\subimg[components/img/celes/64sph_b166nm_l83nm_30deg_far]{0.4\textwidth}
    \caption{$10$-th harmonic scattering by multiple clusters. $b = 2\lambda_{10}$; a, b --- normal incidence; c, d --- incidence at $45^{\circ}$ angle; a, c --- near-field; b, d --- far-field.}
    \label{multi_sph_b2:image}
\end{figure}

\begin{figure}[H]
    (a)\:\subimg[components/img/celes/64sph_b249nm_l83nm_0deg_near]{0.4\textwidth}
    (b)\:\subimg[components/img/celes/64sph_b249nm_l83nm_0deg_far]{0.4\textwidth}
    \\(c)\:\subimg[components/img/celes/64sph_b249nm_l83nm_30deg_near]{0.4\textwidth}
    (d)\:\subimg[components/img/celes/64sph_b249nm_l83nm_30deg_far]{0.4\textwidth}
    \caption{$10$-th harmonic scattering by multiple clusters. $b = 3\lambda_{10}$; a, b --- normal incidence; c, d --- incidence at $45^{\circ}$ angle; a, c --- near-field; b, d --- far-field.}
    \label{multi_sph_b3:image}
\end{figure}
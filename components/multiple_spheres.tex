\section{Multiple clusters}

\noindent(- - 19 - -)\\~\\

And we have results for simple cubic lattice scattering.

\noindent(- - 20 - -)\\~\\
It's described by folowing parameters: node radius $a$, number of nodes at the edge and unit cell length $b$. Field parameters are the same as for single cluster.

\noindent(- - 21 - -)\\~\\
Two cases was considered --- with $b$ equals to single wavelength and $b$ equals to triple wavelength.

For the case of single wavelength we can see efficient scattering by facets of the spatial lattice. Most of the field is localized in the area of clusters.

\noindent(- - 22- -)\\~\\
For triple wavelength due to the increased rarefaction between clusters allows us to get rid of strong reflection, which can be seen from the far-field.


It was shown that using the linear Mie theory, it's possible to quantify the resonance parameters of single cluster and predict potential scattering directions for multiple clusters. 

Using the example of a cubic lattice location of high-density plasma spheres, was calculated options for diffraction control of the tenth laser harmonic for various of the lattice constants and revealed the features of scattering with respect to the angle incidence of radiation on the grating.

The results show the ability to control high harmonics of laser radiation in XUV range using an ionized cluster gas.

\noindent(- - 24 - -)\\~\\

Thanks for your attention.
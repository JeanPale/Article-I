\section{Одиночный кластер}

В рамках теории рассеяния Ми известно, что амплитуду поля вблизи поверхности мишени можно значительно усилить. Для проверки этого было посчитаны значения $m $, отвечающие ранее рассмотренным условиям $\lambda = \lambda_{10}$, $ka = 0.5,\:0.7$, которые оказались равны $m_{0.5} = 1.635i$, $m_{0.7} = 1.851i$.

Были вычислены дальние и ближние результирующие электрические поля для этих двух случаев при $\lambda = \lambda_{L}$ и $\lambda = \lambda_{10}$ с целью сравнения профилей и амплитуды. Видно, что рассеяние первой гармоники в обоих случаях очень близко к рэлеевскому (\autoref{1h_ka0.5:image} б, \ref{1h_ka0.7:image} б) - профиль плоской падающей волны практически не изменяется. Также рэлеевской зависимости \cite{boren_huffman} соответствуют и индикатриссы рассеяния в плоскости поляризации (косинусоидальная зависимость) (\autoref{ka0.5_far_field:image}, \ref{ka0.7_far_field:image}). Совсем другая ситуация в случае $\lambda = \lambda_{10}$ -- профиль волны искажен в результате рассеяния и становится похож на расходящуюся сферическую волну (\autoref{10h_ka0.5:image} б, \ref{10h_ka0.7:image} б). Амплитуда поля в окрестности рассеивающего кластера выше, чем при $\lambda = \lambda_{L}$ (примерно в 5 раз для обоих случаев) (\autoref{10h_ka0.5:image} а, \ref{10h_ka0.7:image} а), также индикатриссы рассеяния в плоскости поляризации имеют б\'{о}льшие амплитуды, что говорит о более эффективном рассеянии в дальнем поле (\autoref{ka0.5_far_field:image}, \ref{ka0.7_far_field:image}). Также наблюдается усиление рассеяния под углами $\theta \approx 0^{\circ},\:60^{\circ},\:-60^{\circ}$ относительно направления распространения плоской волны.

Случай $ka = 0.7$ был также сравнён с аналогичной ситуацией для одиночного наноцилиндра \cite{andreev_lecz} (\autoref{10h_ka0.7:image}в). Видно, что картины поля похожи, в том числе и область локализованного поля в направлении рассеяния на угол $0^{\circ}$ относительно направления распространения плоской волны.

    \begin{figure}[H]
        (а)\:\subimg[components/img/mph/830nm_ka0.5_near]{0.42\textwidth}
        (б)\:\subimg[components/img/mph/830nm_ka0.5_far]{0.42\textwidth}
        \caption{\textbf{Рассеяние лазерной гармоники на одиночном кластере.} $\lambda = \lambda_{L}$, $a \approx 6.4$~нм ($ka = 0.5$); построена $|\vectbf{E}{}|$ в плоскости поляризации, ближнее поле (а) и дальнее (б).}
        \label{1h_ka0.5:image}
    \end{figure}

    \begin{figure}[H]
        (а)\:\subimg[components/img/mph/83nm_ka0.5_near_k_broken]{0.42\textwidth}
        (б)\:\subimg[components/img/mph/83nm_ka0.5_far_k_broken]{0.42\textwidth}
        \caption{\textbf{Рассеяние десятой гармоники на одиночном кластере.} $\lambda = \lambda_{10}$, $a \approx 6.4$~нм ($ka = 0.5$); построена $|\vectbf{E}{}|$ в плоскости поляризации, ближнее поле (а) и дальнее (б).}
        \label{10h_ka0.5:image}
    \end{figure}

    \begin{figure}[H]
        (а)\:\subimg[components/img/mph/830nm_ka0.7_near]{0.42\textwidth}
        (б)\:\subimg[components/img/mph/830nm_ka0.7_far]{0.42\textwidth}
        \caption{\textbf{Рассеяние лазерной гармоники на одиночном кластере.} $\lambda = \lambda_{L}$, $a \approx 8.9$~нм ($ka = 0.7$); построена $|\vectbf{E}{}|$ в плоскости поляризации, ближнее поле (а) и дальнее (б).}
        \label{1h_ka0.7:image}
    \end{figure}

    \begin{figure}[H]
        (а)\:\subimg[components/img/mph/83nm_ka0.7_near_k_broken]{0.42\textwidth}
        (б)\:\subimg[components/img/mph/83nm_ka0.7_far_k_broken]{0.42\textwidth}
        \\(в)\:\subimg[components/img/external/oe-28_screen]{0.46\textwidth}
        \caption{\textbf{Рассеяние десятой гармоники на одиночном кластере.} $\lambda = \lambda_{10}$, $a \approx 8.9$~нм ($ka = 0.7$); построена $|\vectbf{E}{}|$ в плоскости поляризации, ближнее поле (а) и дальнее (б). Также для качественного сравнения приведено поле, рассеянное одиночным наноцилиндром \cite{andreev_lecz} с аналогичным значением $ka$ (в) - здесь волна распространяется справа налево (вдоль отрицательного направления оси $x$ на графике), поляризация вдоль $y$.}
        \label{10h_ka0.7:image}
    \end{figure}

    \img[components/img/mph/ka0.5_far_field]{\textbf{Индикатриссы рассеяния одиночным кластером.} $a \approx 6.4$~нм ($ka = 0.5$).}{ka0.5_far_field:image}{}{0.9\textwidth}

    \img[components/img/mph/ka0.7_far_field]{\textbf{Индикатриссы рассеяния одиночным кластером.} $a \approx 8.9$~нм ($ka = 0.7$).}{ka0.7_far_field:image}{}{0.9\textwidth}

% По полученным резульатам видно, что амплитуда ближнего электрического поля частицы в случае 10-ой гармоники (\autoref{10th_harm:image}) относительно внешнего поля значительно превышает таковую для 1-ой гармоники (\autoref{1st_harm:image}), что можно заметить по контрасту амплитуд вблизи и вдали от рассеивающего объекта.
\section{Single cluster}

\noindent(- - 13 - -)\\~\\

Next we will consider several scattering computational experiments.

\noindent(- - 14 - -)\\~\\
Within the Mie theory, it is well-known that we can significantly enhance the field amplitude near the target. To check this, we will consider first and tenth laser harmonics in two cases of cluster radius: 0.5 and 0.7.

\noindent(- - 15 - -)\\~\\
Total near- and far-field was calculated to compare their amplitudes and scattering profiles. We can see, that the scattering of the laser harmonic (first harmonic) is very close to Rayleigh scattering --- the incident plane wave profile almost does not change. The near-field amplitude value maximum is about fourteen.

\noindent(- - 16 - -)\\~\\
There is completely different situation for 10th harmonic --- the incident plane wave profile is distorted as a result of scattering and becomes like a diverging spherical wave. The near-field amplitude is about 5 times higher than for first harmonic.

\noindent(- - 17 - -)\\~\\
Here is similar situation as for previously considered case with first harmonic - almost Rayleigh scattering without profile distortion and smaller maximum of the field amplitude.

\noindent(- - 18 - -)\\~\\
This case compared with similar situation for scattering by a single nanocylinder. We can see, that field distributions are similar include spherical outgoing far-field wave and localized near-field area in $0^{\circ}$ scattering direction relative to the direction of the incident wave propagation.
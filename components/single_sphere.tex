\section{Одиночный кластер}

    \begin{figure}[H]
        \subimg[components/img/mph/830nm_6.37nm_sphere_n0_k1.73]{0.6\textwidth}
        \subimg[components/img/mph/830nm_6.37nm_sphere_n0_k1.73_log]{0.33\textwidth}
        \caption{\textbf{Рассеяние первой гармоники на одиночном кластере.} $\lambda_1 = \lambda_L = 830$ нм, $a \approx 6.4$ нм;~слева $|\vectbf{E}{}|$ в плоскости поляризации, ближнее поле рассеивающего объекта;~справа $\ln^2 ( |\vectbf{E}{}|^2 + 1)$, дальнее поле.}
        \label{1st_harm:image}
    \end{figure}

    \begin{figure}[H]
        \subimg[components/img/mph/83nm_6.37nm_sphere_n0_k1.73]{0.6\textwidth}
        \subimg[components/img/mph/83nm_6.37nm_sphere_n0_k1.73_log]{0.35\textwidth}
        \caption{\textbf{Рассеяние десятой гармоники на одиночном кластере.} $\lambda_{10} = 83$ нм, $a \approx 6.4$ нм;~слева $|\vectbf{E}{}|$ в плоскости поляризации, ближнее поле рассеивающего объекта;~справа $\ln ( |\vectbf{E}{}|^2 + 1)$, дальнее поле.}
        \label{10th_harm:image}
    \end{figure}

По полученным резульатам видно, что амплитуда ближнего электрического поля частицы в случае 10-ой гармоники (\autoref{10th_harm:image}) относительно внешнего поля значительно превышает таковую для 1-ой гармоники (\autoref{1st_harm:image}), что можно заметить по контрасту амплитуд вблизи и вдали от рассеивающего объекта.
\pagestyle{empty}
\begin{center}
    \MakeUppercase{РЕЦЕНЗИЯ}

    На статью:

    \textit{<<Dynamical phase transition in the open Dicke model>>}

    Авторы:

    J. Klinder, H. Keßler, M. Wolke, L. Mathey and A. Hemmeric
\end{center}

\vfill

В данной работе исследуется динамический фазовый переход в открытой модели Дике, смоделированный в системе атом-полость, находящейся вблизи нулевой температуры. Модель Дике, в данной работе, реализована в режиме слабой диссипации за счет связи конденсата Бозе-Эйнштейна с оптическим резонатором со сверхузкой полосой пропускания. Авторами исследованы динамические критические свойства фазового перехода Хеппа-Либа-Дике.

Авторами была предложена схема эксперимента, основным нововведением которой, является использование резонатора со сверхузкой полосой пропускания, порядка частоты однофотонной отдачи. Используя эту схему, исследователи смогли динамически получить доступ к неадиабатическому режиму.

В результате эксперимента, был обнаружен гистерезис при переходе между нормальной фазой и самоорганизованной коллективной фазой с областью замкнутой петли, показывающей масштабирование степенного закона по отношению к времени гашения.

Данная статья четко структурирована и связанна, экспериментальная установка была выполнена на высоком уровне, в результате этого все теоретические зависимости были подтверждены. Статья изобилует графиками и изображениями, что позволяет лучше понимать изложенный материал. Существенных замечаний по работе нет.

Данная работа позволяет по-новому взглянуть на неравновесную физику системы многих тел с бесконечным диапазоном взаимодействий.

По мнению рецензента, данная работа заслуживает публикации в издании «PNAS»


\vfill\vfill\vfill

Бакалавр физико-математических наук \qquad\qquad\qquad\qquad\qquad\qquad\qquad Л.А.Литвинов

\vfill
\vfill


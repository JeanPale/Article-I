\noindent(- - 2 - -)\\~\\

\section{Introduction}

\noindent(- - 3 - -)\\~\\
Limited size targets interacting with high-intensity coherent radiation is well-studied phenomenon of linear excited surface plasmonic oscillations. Absorption and scattering of incident light in this case good described with Mie theory predicting exist of resonance corresponding to multipole oscillations of part of the target free electrons relative to positive charged ions. In resonance mode efficient exciting of surface plasmons can lead to significant boost internal and external field on fundamental cluster frequency (eigenfrequency). In turn, this can cause enhancement of field scattered on large angles relative to the direction of incident wave. 

\noindent(- - 4 - -)\\~\\
In micrometer wavelengths photon crystals and lattices can be used for direction or diffraction electromagnetic waves~\cite{lin_zhang}, while for x-ray radiation it is possible to use real crystals with regularly placed scattering centers (atoms) with distance of few nanometers~\cite{batterman_cole}. At the same time, large interval between these wavelength orders named XUV (extreme-ultraviolet) is hard to manipulate.

\noindent(- - 5 - -)\\~\\
Within the present work we consider the possibility of directed scattering of short wavelength radiation in the XUV range by scattering on suitable spherical clusters. Similar case with cylindrical symmetry (arrays of nanocylinders as scatterers) was researched earlier~\cite{andreev_lecz}. Of course, nanocylinders are more suitable regarding the control of size an distance parameters at the target manufacturing stage, but arrays of spherical clusters can make possible to manipulate with light direction in three-dimensional space and give a more optimal spatial configuration.

\noindent(- - 6 - -)\\~\\
It is known that a short intense laser pulse can generate high-order harmonics by interacting with dense solid surfaces. But intensity of high-order harmonics generated in gases is at least 4 orders of magnitude less that is not enough to ionize the target and generate a plasma with fully imaginary refractive index that we need --- in our case, spherical clusters are ionized cluster gas (Figure 1). To solve this problem we propose to use intense preceding pulse to pre-ionize the target and reach required plasma generation.

\noindent(- - 7 - -)\\~\\
Common interaction scheme is shown in Figure 2. Harmonics in the main pulse have different intensity depending on the angle, that leads to the angle dependence of output radiance shape. The scattering by a single cluster can be completely described in spherical symmetry and the interaction can be easily modeled with the help of particle in cell simulations. We propose to use linear approximation by Mie theory as assessment for further modeling. In general, we concentrate on a theoretical investigation, supported by simulations, and we point out the applicability for experimental realisation.

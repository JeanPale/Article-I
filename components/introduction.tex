\section*{Введение}

Мишени конечного размера, взаимодействующие с высокоинтенсивным когерентным излучением представляют собой хорошо изученное явление линейно возбужденных 
поверхностных плазмонных колебаний. Поглощение и рассеяние падающего света в таком случае с хорошей точностью могут быть описаны при помощи теории Ми, которая предсказывает существование резонанса, соответствующего мультипольным колебаниям части свободных электронов мишени относительно положительно заряженных ионов. В режиме резонанса эффективное возбуждение поверхностных плазмонов может привести к значительному усилению внутреннего и внешнего 
поля на собственной частоте <мишени? кластера?>. Что может привести к усилению поля, рассеянного на большие углы относительно исходного направления падающей волны.

Известно, что при помощи короткого интенсивного лазерного импульса можно генерировать лазерные гармоники высокого порядка при взаимодействии с плотными твердыми поверхностями \cite{teubner_gibbon_hoh}. 

\addimg[components/img/plasma_area1]{...}{plasma_area1:image}{}{0.8\textwidth}
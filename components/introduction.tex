\section{Введение}

Мишени конечного размера, взаимодействующие с высокоинтенсивным когерентным излучением представляют собой хорошо изученное явление линейно возбужденных 
поверхностных плазмонных колебаний. Поглощение и рассеяние падающего света в таком случае с хорошей точностью могут быть описаны при помощи теории Ми, которая предсказывает существование резонанса, соответствующего мультипольным колебаниям части свободных электронов мишени относительно положительно заряженных ионов. В режиме резонанса эффективное возбуждение поверхностных плазмонов может привести к значительному усилению внутреннего и внешнего 
поля на собственной частоте кластера. Что может привести к усилению поля, рассеянного на большие углы относительно исходного направления падающей волны \cite{andreev_lecz}.

Известно, что при помощи короткого интенсивного лазерного импульса можно генерировать лазерные гармоники высокого порядка при взаимодействии с плотными твердыми поверхностями \cite{teubner_gibbon_hoh}. Эффективность такой генерации... \textit{(нужно ли писать это? т.к. конкретного материала ещё нет...)}. В случае гармоник высокого порядка, генерируемых в газах, интенсивность излучения как минимум на 4 порядка ниже, чего недостаточно для ионизации мишени и генерации плазмы с полностью мнимым показателем преломления. Для решения подобной проблемы предлагается использовать предимпульс с целью предварительной ионизации мишени и достижения генерации в заданых условиях. 

В пределах длин волн порядка микрометра могут быть использованы фотонные кристаллы и решетки для направления или дифракции электромагнитных волн \cite{lin_zhang}, в то время как для подобных манипуляций с рентгеновским излучением могут быть использованы кристаллы с атомами, регулярно \textit{(периодически?)} расположенными на расстоянии нескольких нанометров, в качестве рассеивающих центров \cite{batterman_cole}. При этом большой промежуток между этими диапазонами длин волн, называющийся XUV (extreme-ultraviolet), оказывается трудно манипулируемым.

В данной работе предлагается использование массивов сферических нанокластеров для направленного рассеяния излучения в XUV диапазоне. Обобщенная схема взаимодействия приведена на \autoref{plasma_area1:image}. Гармоники, которые содержит основной импульс, обладают различной интенсивностью под различными углами, что приводит к угловой зависимости формы выходного излучения.

Рассеяние одиночным сферическим кластером с хорошей точностью описывается в рамках теории Ми, поэтому такое линейное приближение предлагается использовать и в случае множества кластеров с целью качественной оценки и дальнейшего уточнения при помощи particle-in-cell (PIC) моделирования.

\img[components/img/plasma_area1]{\textbf{Схема взаимодействия.} Плоскость поляризации параллельна одной из граней кубической области. Размеры сферических кластеров порядка единиц нанометров, расстояния между ними не менее сотни нанометров. Распределение кластеров внутри кубической области в общем случае произвольно, кластеры не пересекают грани области.}{plasma_area1:image}{}{0.8\textwidth}

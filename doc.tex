% ------------------------- %
%  Layout
% ------------------------- %

\documentclass[11pt]{article}
\linespread{1.05}
\setlength{\parskip}{0.45em}

% ------------------------- %
%  Packages
% ------------------------- %

\usepackage[utf8]{inputenc}
\usepackage[T2A]{fontenc}
\usepackage[russian, english]{babel}
\usepackage{fancyhdr}
\usepackage{listings}
\usepackage{booktabs}
\usepackage{xcolor}
\usepackage[margin=1.0in, headsep=0.3in]{geometry} % margin=0.9
\usepackage{xspace}
\usepackage{graphicx}
\usepackage[font=small, labelfont=bf, width=0.95\linewidth]{caption}
\usepackage{floatrow}
\usepackage{amsmath}
\usepackage{fouriernc}
\usepackage{MnSymbol}
\usepackage{tempora}
\usepackage{mathtools}
\usepackage[unicode=true,hidelinks]{hyperref}
\usepackage[fixlanguage]{babelbib}
\usepackage[oldsyntax]{stackengine}

\def\stacktype{S}\Sstackgap=-4.3pt

% ------------------------- %
%  Assets
% ------------------------- %

\def \w{\omega}

% ------------------------- %

\def \eq{\begin{equation}}
\def \qe{\end{equation}}
\def \eqc{\begin{equation*}}
\def \cqe{\end{equation*}}

% ------------------------- %
\newcommand{\symbf}[2][bold symbol]{
    \mbox{\boldmath${#2}$}
}

% ------------------------- %

\newcommand{\indexbf}[2][bold index]{
    \mbox{\boldmath{#2}}
}

% ------------------------- %

\newcommand\shifthat[2]{%
  \stackengine{\Sstackgap}{$#2$}{\(\hspace{#1}\hat{}\)}{O}{l}{F}{T}{S}}

% ------------------------- %

\newcommand{\operator}[2][operator]{
    % \symbf{\hat{#2}} % works with default font
    % \hat{\textit{\textbf{#2}}} % italic bold style
    % bold style
    \if H#2\shifthat{0.5em}{#2}\else
    \if d#2\shifthat{0.49em}{#2}\else
    \if q#2\shifthat{0.35em}{#2}\else
    \if \mu#2\shifthat{0.35em}{#2}\else
    \shifthat{0.45em}{#2}
    \fi
    \fi
    \fi
    \fi
}

% ------------------------- %

\newcommand{\vectoperator}[2][operator]{
    % \symbf{\hat{#2}} % works with default font
    % \hat{\textit{\textbf{#2}}} % italic bold style
    % \symbf{\hat{#2}} % works with default font
    % \hat{\textit{\textbf{#2}}} % italic bold style
    % bold style
    \if d#2\shifthat{0.367em}{\textbf{#2}}\else
    \if m#2\shifthat{0.4em}{\textbf{#2}}\else
    \shifthat{0.275em}{\textbf{#2}}
    \fi
    \fi
}

% ------------------------- %

\newcommand{\vect}[3][vector]{
    \overrightarrow{#2_{#3}}
}

% ------------------------- %

\newcommand{\vectbf}[2][bold vector]{
    % \vect{\textit{\textbf{#2}}}
    \vect{\textbf{#2}}
}

% ------------------------- %

\newcommand{\pd}[3][empty]{
    \frac{\partial {#2}}{\partial {#3}}
}

% ------------------------- %

\newcommand{\func}[5][empty]{
    {#2}_{#3}^{#4} \left( {#5} \right)
}

% ------------------------- %

\newcommand{\underrel}[3][]{
    \mathrel{\mathop{#3}\limits_{
        \ifx c#1\relax\mathclap{#2}\else#2\fi
    }}
}

% ------------------------- %

\newcommand{\addimg}[5][anything]{
    \begin{figure}[H]{
        \center{\includegraphics[width={#5}]{{#1}} {#4}} % 0.8\textwidth
        \caption{#2}
        \label{#3}}
    \end{figure}
}

% ------------------------- %
%  Document
% ------------------------- %

\begin{document}

	% declare specific options

	\selectlanguage{russian}

	\DeclareGraphicsExtensions{.pdf,.png,.jpg}

	%\pagestyle{fancy}
	\pagestyle{empty}
	\fancyhf{}
	\fancyhead[L]{\textit{\nouppercase{\leftmark}}}
	\fancyfoot[C]{\thepage}

	\thinmuskip = 1mu
	\thickmuskip = 6mu

	\renewcommand{\equationautorefname}{}

	% add components

	% \title{Efficient directed scattering of XUV radiation using high-density spherical clusters}
	% \author{}
	% \maketitle

	\noindent(- - 1 - -)\\~\\
	Здравствуйте, меня зовут Литвинов Лев, и я хотел бы представить вам свою работу под названием <<Эффективное направленное рассеяние жесткого ультрафиолетового излучения при помощи высокоплотных сферических кластеров>>. Мой научный руководитель --- Андреев Александр Алексеевич.

	\section{Introduction}

Limited size targets interacting with high-intensity coherent radiation is well-studied phenomenon of linear excited surface plasmonic oscillations. Absorption and scattering of incident light in this case good described with Mie theory predicting exist of resonance corresponding to multipole oscillations of part of the target free electrons relative to positive charged ions. In resonance mode efficient exciting of surface plasmons can lead to significant boost internal and external field on fundamental cluster frequency (eigenfrequency). In turn, this can cause enhancement of field scattered on large angles relative to the direction of incident wave. 

In micrometer wavelengths photon crystals and lattices can be used for direction or diffraction electromagnetic waves~\cite{lin_zhang}, while for x-ray radiation it is possible to use real crystals with regularly placed scattering centers (atoms) with distance of few nanometers~\cite{batterman_cole}. At the same time, large interval between these wavelength orders named XUV (extreme-ultraviolet) is hard to manipulate.

Within the present work we consider the possibility of directed scattering of short wavelength radiation in the XUV range by scattering on suitable spherical clusters. Similar case with cylindrical symmetry (arrays of nanocylinders as scatterers) was researched earlier~\cite{andreev_lecz}. Of course, nanocylinders are more suitable regarding the control of size an distance parameters at the target manufacturing stage, but arrays of spherical clusters can make possible to manipulate with light direction in three-dimensional space and give a more optimal spatial configuration.

\img[components/img/cluster_gas_sheme]{Ionized cluster gas generation process.}{cluster_gas_sheme:image}{}{0.7\textwidth}

It is known that a short intense laser pulse can generate high-order harmonics by interacting with dense solid surfaces. But intensity of high-order harmonics generated in gases is at least 4 orders of magnitude less that is not enough to ionize the target and generate a plasma with fully imaginary refractive index that we need --- in our case, spherical clusters are ionized cluster gas (\autoref{cluster_gas_sheme:image}). To solve this problem we propose to use intense preceding pulse to pre-ionize the target and reach required plasma generation.

Common interaction scheme is shown in \autoref{plasma_area1:image}. Harmonics in the main pulse have different intensity depending on the angle, that leads to the angle dependence of output radiance shape. The scattering by a single cluster can be completely described in spherical symmetry and the interaction can be easily modeled with the help of particle in cell simulations. We propose to use linear approximation by Mie theory as assessment for further modeling. In general, we concentrate on a theoretical investigation, supported by simulations, and we point out the applicability for experimental realisation.

\img[components/img/plasma_area2]{Interaction scheme. The plane of polarization is parallel to one of the faces of cubic region. The dimensions of spherical clusters are about a few nanometers, distance between them is at least wavelength. In general, the distribution of clusters within a cubic region is random, clusters do not intersect the edges of the region and each other.}{plasma_area1:image}{}{0.8\textwidth}

	\section{Базовая модель}

Рассмотрим одиночный сферический кластер радиуса $a$, облученный коротким импульсом длительностью $\tau \approx 20$ фс и интенсивностью $I_{h} \approx 10^{14}$ $\textrm{Вт/см}^2$. Модель Друде даёт представление диэлектрической функции плазмы:

    \eq
		\varepsilon (\omega) = 1 - \left( \frac{\omega_{pe}}{\omega} \right)^2 \frac{1}{1 + i \beta_{e}}, \qquad \omega_{pe} = \sqrt{\frac{4 \pi e^2 n_e}{m_e}}
		\label{eps_plasma}
	\qe

Здесь $\omega$ -- рассматриваемая гармоническая частота, $\omega_{pe}$ -- электронная плазменная частота, $e$ и $m_e$ -- заряд и масса электрона, $n_e = Z n_i$ -- электронная плотность, где $Z$ - средняя степень ионизации, $n_i = ? \cdot 10^{22} \textrm{см}^{-3}$ -- ионная плотность \textit{(материал не выбран)}. $\beta_{e} = v_e / \omega$ и $v_e$ коэффициент электрон-ионных столкновений в приближении Спитцера. Так как предполагается рассмотрение рассеяния, плотность кластера должна быть выше критической для заданной частоты $n_c = \omega^2 m_e / 4 \pi e^2$. Тогда, например, для 10 гармоники лазерного излучения с длиной волны $\lambda_{L} = 830$ нм мы получаем условие $n_e > 1.3 \cdot 10^{23} \textrm{см}^{-3}$.

Теория Ми может быть использована для описания упругого рассеяния электромагнитных волн частицами произвольного в случае линейных взаимодействий, а также позволяет получить описание рассеянного поля и поля внутри рассеивающего объекта. Основной шаг - решение скалярного уравнения Гельмгольца в правильной системе координат (в данном случае сферической) и получение векторных решений. Для сферического кластера можем записать решение соответствующего уравнения, используя сферические функции Бесселя и Ханкеля $n$-ого порядка [?].

Возьмем плоскую волну, распространяющуюся вдоль оси $z$ декартовой системы координат, поляризованную вдоль оси $x$, что может быть записано как 

    \eq
        \vectbf{E}{i} = E_0\:e^{i\omega t - ikz}\:\vectbf{e}{x},
        \label{E_i_sph}
    \qe

\noindentгде $k = \omega/c$ - волновое число, $\vectbf{e}{x}$ - единичный вектор оси $x$, также являющийся вектором поляризации (\autoref{single_sph_scheme:image}). 

    \img[components/img/single_sph_scheme]{\textbf{Схема базовой модели.} ...}{single_sph_scheme:image}{}{0.73\textwidth}

Далее эту плоскую можно разложить в ряд, используя обобщённое разложение Фурье. В случае изотропной среды имеем следующий вид рассеянного поля \cite{boren_huffman}:

    \eq
		\vectbf{E}{s} = \sum_{n = 1}^{\infty}E_n \left[ i a_n\left(ka, m\right) \vectbf{N}{}^{(3)}_{e1n} - b_n\left(ka, m\right) \vectbf{M}{}^{(3)}_{o1n} \right], \qquad E_n = i^{n} E_0 \frac{2n + 1}{n \left( n + 1\right)}
        \label{E_s_sph}
	\qe

Здесь $n$ - номер векторной гармоники, полученный после трансформации из декартовой системы координат в сферическую, $m = \sqrt{\varepsilon\left(\omega\right)}$ - коэффициент преломления мишени. Коэффициенты векторных гармоник имеют следующий вид:


    \eq
		a_n(x,\:m) = \frac{m \func{\psi}{n}{\prime}{x} \func{\psi}{n}{}{mx} - \func{\psi}{n}{\prime}{mx} \func{\psi}{n}{}{x}}{m \func{\xi}{n}{\prime}{x} \func{\psi}{n}{}{mx} - \func{\psi}{n}{\prime}{mx} \func{\xi}{n}{}{x}},
		\label{an_bessel}
	\qe

    \eq
        b_n(x,\:m) = \frac{\func{\psi}{n}{\prime}{x} \func{\psi}{n}{}{mx} - m \func{\psi}{n}{\prime}{mx} \func{\psi}{n}{}{x}}{\func{\xi}{n}{\prime}{x} \func{\psi}{n}{}{mx} - m \func{\psi}{n}{\prime}{mx} \func{\xi}{n}{}{x}},
        \label{bn_bessel}
    \qe

\noindentгде $\func{\psi}{n}{}{z} = z \func{j}{n}{}{z}$, $\func{\xi}{n}{}{z} = z \func{h}{n}{}{z}$ -- функции Риккати-Бесселя, $h_n = j_n + i \gamma_n$ -- сферические функции Ханкеля первого рода.

В случае сферической симметрии амплитуда рассеянного поля максимальна для $m^2 = - (n+ 1) / n$ при $ka \ll 1$, что даёт соответствующий набор резонансных плотностей в бесстолкновительном случае $n_e = n_c(2n + 1) / n$. Это можно получить, используя нулевое асимптотическое приближение функций Бесселя, в результате чего коэффициенты (\Autoref{an_bessel, bn_bessel}) значительно упрощаются:

    \eq
        a_n\left( x \to 0,\:m \right) = \left( 1 + 2i \frac{ (2n - 1)! (2n + 1)!}{4^n \: n! (n + 1)!} \frac{\left(m^2 + \frac{n + 1}{n} \right)}{(m^2 - 1)} \frac{1}{x^{2n+1}} \right)^{-1}, \qquad b_n\left( x \to 0,\:m \right) = 0
        \label{ab_asymp}
    \qe

Такое приближение можно использовать вместо (\Autoref{an_bessel, bn_bessel}) для объектов достаточно маленького радиуса, но уже при $ka \sim 1$ оно перестаёт быть разумным, особенно для больших $n$. Вместо него в таком случае лучше подходит аппроксимация первого порядка:

    \eq
		a_n\left( x ,\:m \right) = \left( 1 + i \frac{ C_n x^{-1 -2n} \left( (4(1 + n + m^2 n) (-3 + 4n (1 + n)) - 2(m^2 - 1)(3 + n(5 + 2n + m^2 (2n - 1))) x^2) \right)}{\pi (m^2 - 1)(2n + 3)(n + 1)(4(2n + 3) - 2(m^2 + 1)x^2)} \right)^{-1}
		\label{an_sph_asymp1}
	\qe
	\eqc
		C_n = 2^{1 + 2n} \Gamma(n - \frac{1}{2}) \Gamma(n + \frac{5}{2})
	\cqe

На \autoref{ab_asymp:image} показана зависимость коэффициента рассеяния от электронной плотности для двух различных значений радиуса в рамках нулевого асимптотического приближения. Видно, что с ростом $n$ ширина резонансного пика быстро уменьшается, а также б\'{о}льшим радиусам (безразмерным) $ka$ соответствует б\'{о}льшая их ширина. Помимо этого, с ростом радиуса растет и значение резонансной плотности, что видно на \autoref{nenc_123:image}.

    \begin{figure}[ht]
		\subimg[components/img/sph_base/sph_ka0.5_123]{0.72\textwidth}
		\subimg[components/img/sph_base/sph_ka1.5_123]{0.72\textwidth}
		\caption{\textbf{Коэффициенты сферических гармоник.} Верхний рисунок для $ka = 0.5$, нижний для $ka = 1.5$. Коэффициенты отвечают гармоникам $\vectbf{N}{}^{(3)}_{e1n}$, $\beta_e = 0$.}
		\label{ab_asymp:image}
	\end{figure}

    \img[components/img/sph_base/nenc_123]{\textbf{Резонансная электронная плотность в зависимости от радиуса.} Кривые посчитаны в точках максимума коэффициента (\ref{an_sph_asymp1}), $\beta_e = 0$.}{nenc_123:image}{}{0.73\textwidth}

Для того, чтобы обеспечить качественное рассеяние для $\lambda_10 = \lambda_L / 10 = 83$ нм, необходимо взять электронную плотность около $n_e \approx ... $ в случае $ka \approx 0.5$. 



	\section{Одиночный кластер}
	\section{Множество кластеров в рамках кубической области}

	% add bibliography
	% \selectbiblanguage{english}
	% \bibliographystyle{ieeetr}
	% \bibliography{components/bibliography.bib}

\end{document}